\documentclass[UTF8]{ctexart}

\usepackage[linesnumbered,boxed,ruled,commentsnumbered]{algorithm2e}
\usepackage{bm}
\usepackage{graphicx}
\usepackage{float}
\usepackage[bookmarks=true]{hyperref}
\usepackage{amsmath}
\usepackage{geometry}


\geometry{a4paper,left=3cm,right=3cm,top=4cm,bottom=4cm}

\begin{document}
\title{$sin(x)$的数值计算与误差分析}
\author{陈昭熹 2017011552}
\maketitle
\tableofcontents
\newpage

\section{引言}
本文实现了通过数值方法求$sin(x)$,在一开头给出本文所实现的全部算法流程简述:
\begin{itemize}
    \item[\textbf{逼近法}] 使用级数逼近$sin(x)$
    \item[\textbf{常微分方程法}]通过三角函数关系$\sin^2{x}+\cos^2{x}=1$,利用改进欧拉法求解关于$sin(x)$的微分方程。
    \item[\textbf{方程求根法}] 利用函数逼近$\arcsin{y}$,再利用牛顿法求解$x=\arcsin{y}$得到$y$,即$sin(x)$的值。
\end{itemize}

后面章节中,逐节介绍各个算法的原理、误差分析、算法流程以及计算代价和收敛速度。第5节介绍算法利用C++的程序实现以及具体流程(以流程图形式给出),第6节展示一些实验结果并做总结。

\section{逼近法}

本节介绍利用逼近法计算$sin(x)$的任意精度算法。

\subsection{算法原理}

众所周知,根据$sinx$在$x_0=0$的邻域$(-h+x_0,h+x_0)$内的Taylor展开式,有如下式子成立:
\begin{equation}
    \sin{x} = \sum^\infty_{n=0} \frac{(-1)^nx^{2n+1}}{(2n+1)!}=x-\frac{x^3}{3!}+\frac{x^5}{5!}-\frac{x^7}{7!} \dots + \frac{x^{2n+1}}{(2n+1)!}+R_k(x)
\end{equation}

注意这里写了余项形式(为了表示方便,将展开阶次表示为k,有k=2n+1),因此可以精确取等。若存在正实数$M_k$使得区间$(-h,h)$上的任意x均有$|f^{(k+1)}(x)|\leq M_k$,则上式中余项估计为:
\begin{equation}
    |R_k(x)|\leq M_k \frac{h^{k+1}}{(k+1)!}
\end{equation}

这样的一个上界估计对$x_0$的邻域内的任意x均成立,是一个一致估计。利用式(1),只需要控制展开的阶数,就可以实现任意精度的$sinx$数值计算。

需要注意的是,这里有一个前提,即上述展开是在原点处的邻域内进行的,因此需要\textbf{通过$sinx$的周期性,尽可能将自变量x变换到原点附近},避免不满足邻域条件导致误差过大。


\subsection{误差分析}

\subsubsection{方法误差}

上一小节已给出逼近法的余项形式,这里进行误差分析。由$sinx$无穷阶光滑特性及周期性可以得到$$M_k \leq 1$$
因此有方法误差
\begin{equation}
    |R_k(x)| \leq \frac{h^{k+1}}{(k+1)!}
\end{equation}

事实上,这里可以通过微分中值定理得到一个精确地误差表达形式,即用$\sin^{(k+1)}{(\xi)}$来代替$M_k$,但是这样无法进行量化分析,因此进行一定的放缩,给出上界。式子中的$h$即实现时候将所有x值利用周期性映射到原点附近的区间$(-h,h)$,一般为$(-\pi/2,\pi/2)$。依据这个上界,可以通过控制多项式展开的阶数$k$来控制方法误差,在理论上达到任意精度。

值得注意的是对于偶数位精度要求,本方法具备天然高一阶的精度(即k=2n+1)。

\subsubsection{舍入误差}

假设每个阶次的计算均精确到$d$位小数,则存储带来的舍入误差为:$$\delta_0 = \frac{1}{2}\times 10^{-d}$$

则求和带来的舍入误差为:
\begin{equation}
    \delta = (n+1) \cdot \delta_0 = (n+1) \times \frac{1}{2}\times 10^{-d}
\end{equation}

\subsubsection{总误差}

因此总误差
\begin{equation}
    A = |R_k(x)| + \delta \leq \frac{h^{k+1}}{(k+1)!} + \frac{n+1}{2} \times 10^{-d}
\end{equation}

\subsection{算法流程}

\subsection{计算代价与收敛速度}
计算代价方面,由于使用递归算法,因此计算阶乘和幂级数的代价均为$O(k)$,而求和的代价也为$O(k)$,因此总体来说还是一个线性的计算速度$O(3k)$。由于余项收敛是泰勒展开成立的条件,因此无需担心收敛速度慢的问题,因为阶乘的增长速度远比幂级数快得多,严格计算则需要归约$\frac{h^{k+1}}{(k+1)!}$,这个数学问题显然比本问题复杂得多。值得注意的是,当$k+1>h$时,$|R_k(x)|$开始以$O(a^{k}),0<a<1$的速度收敛,因此整体的收敛速度仍是常数$O(h)$。这也从另一个侧面说明,在计算过程中将自变量x变换到较小的原点邻域内的巨大作用,若使用原始的x进行计算,会使得h值过大,导致在相同精度要求下,收敛速度急剧变慢,计算代价升高。
\section{常微分方程法}

本节介绍利用常微分方程法计算$sinx$的任意精度算法。

\subsection{算法原理}
利用$\frac{dsin(x)}{dx}=cos(x)$,可以得到下面的常微分方程组:
\begin{equation}
    \frac{d}{dx}\left(
    \begin{matrix}
            cosx \\
            sinx
        \end{matrix}
    \right)=\left(
    \begin{matrix}
            -sinx \\
            cosx
        \end{matrix}
    \right)
\end{equation}

尽管需要计算$sinx$的数值,但还是存在可以加以利用的先验知识——$sin0 = 0$,利用这一点的值作为初始条件,采用改进欧拉法即可求解这一微分方程组在给定点的解。

改进欧拉法分为两个步骤:预测和矫正,若选择的区间间隔为$h$,则可以表达为下式:

\begin{equation}
    \begin{cases}
        \bar{y}_{n+1}=y_n+hf(x_n,y_n) \\
        y_{n+1}=y_n+\frac{h}{2}[f(x_n,y_n)+f(x_{n+1},\bar{y}_{n+1})]
    \end{cases}
\end{equation}

对于本问题,有初始条件:
\begin{equation}
    x_0=0,cos(x_0)=1,sin(x_0)=0
\end{equation}

虽然改进欧拉法只给出了求解一个方程的步骤,注意到本问题的微分方程组有良好的相关性,可以“串联”在一起,就可以较好的将改进欧拉法推广到上面。为表达方便将$cos(x)$简写为$c$,将$sin(x)$简写为$s$,下标表示与$x$的含义相同,则具体做法如下:
\begin{equation}
    \begin{aligned}
        \bar{c}_{n+1} &= c_n + h\times(-s_n)\\
        \bar{s}_{n+1} &= s_n + h\times c_n\\
        c_{n+1} &= c_n + \frac{h}{2}[(-s_n)+(-\bar{s}_{n+1})]\\
        s_{n+1} &= s_n + \frac{h}{2}(c_n+\bar{c}_{n+1})
    \end{aligned}
\end{equation}

\subsection{误差分析}

\subsubsection{方法误差}


\subsubsection{舍入误差}

\subsubsection{总误差}

\subsection{算法流程}

\subsection{计算代价与收敛速度}

\section{方程求根法}

\subsection{算法原理}

\subsection{误差分析}

\subsection{算法流程}

\subsection{计算代价与收敛速度}

\section{算法流程与实现}

\section{结果与总结}
\end{document}